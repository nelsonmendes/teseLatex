\chapter{Introduction} \label{chap:intro}

\section*{}

This chapter presents the context and motivation of this thesis, describing the main goals, its objectives and the expected results.

\section{Context and motivation} \label{sec:context}

Nowadays current markets are changing, we can see more often the globalization phenomenon and with that organizations are compelled to streamline their business in order to achieve a favorable market position and be able to maintain or increase their competitiveness.

In our everyday lives software takes an important role, it is everywhere and is needed more often. When is in development it is important to make it more efficiently and with more quality. For organizations that have software currently in development failures and errors are not allowed and each one of them implies increased costs and resources being wasted. To avoid this scenario and to achieve maximum efficiency and agility, their processes and their methodologies need to be less time consuming and more effortless so good practices need to be followed in order to allow them focus on what really matters: value creation. This will provide them advantages and make them more trustful.

Organizations need to ensure that their products and services consistently meet customer’s requirements, and that quality is consistently improved and certifications are a formal recognition of those ideals. Sadly those recognitions take too much time and effort and in some cases they are very painful and expensive.

Capability Maturity Model Integration(CMMI) is a framework of best practices and does not describe the processes themselves, it describes the characteristics of good processes in order to improve organizations and is required by many U.S. Government contracts, especially in software development.

SCAMPI is the Standard CMMI Appraisal Method for Process Improvement and it provided benchmark quality ratings relative to CMMI models.

SCRAIM is a life cycle and project management tool developed by Stronstep combined with process management techniques. It is going to provide the background and the base to work and simplify those kind of evaluations in order to save time and money. That way companies will deliver their products and services better, faster, and cheaper.

\section{Goals and expected results} \label{sec:goals}

The main goal is of this dissertation is to develop a group of methodologies, techniques and tools integrated in the SCRAIM interface, that will make evaluations and certain parts of certifications easier and less painful for the SCRAIM users.
Although there are a number of life cycle and project management tools, few combine this with process management techniques. SCRAIM combines the two and will provide the users new features that will semi-automate the assessment for certification of an organization. 


%\begin{figure}[h]
%	\begin{center}
%		\leavevmode
%		\includegraphics[width=0.86\textwidth]{thesis_goals}
%		\caption{SCAMPI results}
%		\label{fig:arch}
%	\end{center}
%\end{figure}

As final result, it will be shown how each CMMI practice is evaluated.

The steps to achieve that evaluation are:
\begin{itemize}
	\item Having SCRAIM as the basis for project activity take a sample of projects;
	\item Analyze the project activity in SCRAIM;
	\item Map the information to CMMI;
	\begin{itemize}
		\item Determine what are the good practices presented in SCRAIM, that can be mapped to CMMI;
		\item For each one of them investigate and conclude if that practice is being followed and fully satisfied;
	\end{itemize}
	\item Generate a matrix:
	\begin{itemize}
		\item Each column will represent a good practice that needs to be followed and be satisfied;
	\end{itemize}
	\item Generate the final outputs and evaluations.
\end{itemize}

The full-automated process is not yet feasible, so human intervention is still mandatory. With the use of SCRAIM, good practices will be followed and in the end the generated information will facilitate the decision making process. We can see many advantages of this innovation, and we believe that the application of this innovation will help reducing the costs and time of one evaluation using the SCAMPI method.

\section{Document structure} \label{sec:Structure}

This document is divided into four main chapters. The first and present chapter serves as an introduction where is presented the context and motivation for this thesis and also the goals and expected results to be delivered.

In the chapter 2 is made a problem analysis, giving insight about CMMI, SCAMPI and the tool to be used SCRAIM.

Chapter 3 is a representation of the state of the art and related work for the assessment world. Is described in detail the most used and most important tools that are currently being used in the appraisals.

In the chapter 4 is made a proposal of the envision approach and presented the work plan for the semester.

The final chapter is made a resume of the document and presented the final conclusions about it.