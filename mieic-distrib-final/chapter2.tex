\chapter{Problem analysis} \label{chap:problem}

\section*{}

The evaluation for certification is one complex process, and requires many approaches, some acquired knowledge and some experience. To understand the problem and objectives of this dissertation, it is necessary to understand what is CMMI, in particular the SCAMPI \citep{SCAMPITeam2013} method and what is SCRAIM.

\section{CMMI}

\subsection{What is CMMI}

To understand better what is CMMI \citep{Development2010} we need to understand what is Capability model.

Capability Maturity Models contain essentially elements of effective processes, based on concepts developed by Crosby, Deming, Juran, and Humphrey.

The SEI (The Carnegie Mellon Software Engineering Institute that is a federally funded research and development center headquartered on the campus of Carnegie Mellon University in Pittsburgh, Pennsylvania, United States) adopted the process management premise, "the quality of a system or product is highly influenced by the quality of the process used to develop it and keep it" and defined CMMs that incorporated this premise.

\begin{figure}[h]
	\begin{center}
		\leavevmode
		\includegraphics[width=0.6\textwidth]{CMMI_constelations}
		\caption{History of CMMs}
		\label{fig:historycmmi}
	\end{center}
\end{figure}

%referenciar a imagem no texto de alguma forma

CMMI stands for Capability Maturity Model Integration and is an evolution of CMM like shown in the Figure \ref{fig:historycmmi}.
It is a framework of best practices administered and sold by the Carnegie Mellon University, and for some business activities is required and mandatory like many DOD (United States Department of Defense) and U.S. Government contracts, especially in software development.


The CMMI model does not describe the processes themselves; it describes the characteristics of good processes, thus providing guidelines for companies developing or honing their own sets of processes.

Carnegie Mellon University says that CMMI can be used to guide an organization, a division and process improvement across projects. The CMMI processes and methodologies can be classified according to maturity levels.

Currently CMMI is on Version 1.3 and is registered in the United States Patent and Trademark Office by Carnegie Mellon University.

\subsection{CMMI models and process areas}
Best practices of CMMI are published in documents called models, each model addresses a different area of interest. The current version of CMMI, version 1.3, has three different areas of interest: development \citep{Chrissis2006}, acquisition and services.

These models are produced taking for base the CMMI framework that contains all the goals and practices used to produce the models that are part of CMMI constellations. The CMMI models contain 16 core process areas, they cover basic concepts fundamental to process improvement in any area of interest. 

\begin{figure}[h]
	\begin{center}
		\leavevmode
		\includegraphics[width=0.7\textwidth]{goalsandpractices}
		\caption{Specific and generic goals and practices}
		\label{fig:goalsandpractices}
	\end{center}
\end{figure}


For each process area can be defined a set of goals and practices, the Figure \ref{fig:goalsandpractices} is a diagram where is shown the connection between process areas, goals and practices. 

There are two types of goals and practices:
\begin{itemize}
	\item Generic goals and practices: Part of every process area.
	\item Specific goals and practices: Specific to a given process area.
\end{itemize}

A process area is satisfied only when the company processes cover all specific and generic practices and goals for the process area idealized.

The material in core process areas is almost the same for all constellations of CMMI, the rest of the material need to be adjusted to a specific area of interest, so the material wont be exactly the same.



\subsection{CMMI model framework}
CMMI framework is a basic structure that organizes and groups the CMMI components, elements of the current models, rules, methods for model generations, appraisal methods and training material, contains too process areas that will vary for each one of the CMMI areas that will be used. Process areas are the areas that cover the organization processes.

For the latest version of CMMI for Development (Version 1.3) there are 22 Process Areas, which represents the product aspects and the coverage for the organizational processes.


\subsection{CMMI representations}

CMMI is available in two representations: continuous and staged.

The continuous representation is represented by capability levels. Allows each organization to select the order of improvement that best meets their business objectives or those to which the organization assigns a high degree of risks. Enables comparisons across and among organizations on a process-area by process-area basis.


The staged representation is designed to provide a standard sequence of improvements, by maturity levels, each serving as foundation for the next. This representation results in a single rating (Maturity Level) that summarizes appraisal results and can serve as a basis for comparing the maturity of different projects and organizations.

Each representation has advantages and disadvantages. Staged representation is focused by organizational maturity, continuous representation, by the hand is focused by process are capability.

Organizational maturity and process area capability are similar concepts. The difference between them is that organizational maturity pertains to a set of process areas across an organization, while process area capability deals with a set of processes relating to a single process area or specific practice.

In the pictorial diagram bellow in the Figure \ref{fig:cmmirepresentation} both the presentations are represented where ML represents Maturity Level and PA represents Process Area.

\begin{figure}[h]
	\begin{center}
		\leavevmode
		\includegraphics[width=0.6\textwidth]{cmmirepresentation}
		\caption{CMMI representations}
		\label{fig:cmmirepresentation}
	\end{center}
\end{figure}

\subsection{Maturity levels in CMMI for development}
\begin{figure}[h]
	\begin{center}
		\leavevmode
		\includegraphics[width=0.6\textwidth]{Maturitylevels}
		\caption{CMMI maturity levels}
		\label{fig:maturitylevels}
	\end{center}
\end{figure}
Processes under the CMMI methodology are rated and grouped in maturity levels. There are five  maturity levels defined as: Initial, Managed, Defined, Quantitatively Managed, Optimizing. These maturity levels that are rated are presented and awarded for levels 2 through 5.

%\begin{itemize}
%	\item Maturity Level 2 - Managed
%	\begin{itemize}
%		\item CM - Configuration Management Measurement and Analysis
%		\item PMC - Project Monitoring and Control
%		\item PP - Project Planning
%		\item PPQA - Process and Product Quality Assurance
%		\item REQM - Requirements Management
%		\item SAM - Supplier Agreement Management
%
%	\end{itemize}
%	\item Maturity Level 3 - Defined
%	\begin{itemize}
%		\item DAR - Decision Analysis and Resolution
%		\item IPM - Integrated Project Management
%		\item OPD - Organizational Process Definition
%		\item OPF - Organizational Process Focus
%		\item OT - Organizational Training
%		\item PI - Product Integration
%		\item RD - Requirements Development
%		\item RSKM - Risk Management
%		\item TS - Technical Solution
%		\item VAL - Validation
%		\item VER – Verification	
%	\end{itemize}
%	\item Maturity Level 4 - Quantitatively Managed
%	\begin{itemize}
%		\item OPP - Organizational Process Performance
%		\item QPM - Quantitative Project Management		
%	\end{itemize}
%	\item Maturity Level 5 - Optimizing
%	\begin{itemize}
%	\item CAR - Causal Analysis and Resolution
%	\item OPM - Organizational Performance Management
%	\end{itemize}
%\end{itemize}

\subsection{Capability levels in CMMI for development}

In CMMI models with a continuous representation, there are six capability levels designated by the numbers 0 through 5.

A capability level is a plateau that describes the organization's capability relative to a process area and consists in a group of related specific and generic practices associated with a process area; can improve the organization's processes in that process area.

Capability levels are also cumulative, so a higher capability levels contains the attributes of the lower levels.

\begin{figure}[h]
	\begin{center}
		\leavevmode
		\includegraphics[width=0.8\textwidth]{capability}
		\caption{CMMI maturity levels}
		\label{fig:capabilitylevels}
	\end{center}
\end{figure}


\section{SCAMPI}
Organizations cannot be certified in CMMI, so there is something called appraisal and an organization is appraised.

In an appraisal the organization gets awarded a maturity level from one to five or a capability level achievement profile. As said before many organizations are required to get some kind of recognition and others find value measuring their progress and determining how well the processes adopted by the organization are compared to CMMI best practices, to meet contractual and customers requirements and to know which areas they can improve and appraisals are the right way to do it.

Appraisals using a CMMI model must comply with the requirements set out in the Appraisal Requirements for CMMI (ARC) document. There are three classes of appraisals, A, B and C. All of them compare the processes used in the organization to CMMI processes and best practices, that way is identified improvements to make. From all three classes of appraisals the most formal is class A and it is the only one that can output a level rating.

When an appraisal is done teams use a CMMI model and an ARC document. The results from the teams are used to plan improvements for the organization.

Statistics are made and updated every six months in a maturity profile since the release of CMMI show us that the median times to move from Level 1 to Level 2 is 5 months, and from that to Level 3 more 21 months.

\subsection{What is SCAMPI}
SCAMPI is the abbreviation for Standard CMMI Appraisal Method for Process Improvement and is an appraisal method that meets all the ARC requirements.
In SCAMPI appraisals there are three types of distinct classes: Class A, B and C appraisal methods. The most rigorous method and officially recognized as that is the Class A method and its the only method that can result in a benchmark quality rating. 
%SCAMPI B and C provide organizations improvements less formal than the class A, however still can identify improvements to be done.

Results SCAMPI appraisal can be published on the CMMI web site of SEI, if the organizations approves this. This appraisal supports ISO/IEC 15504, Software Process Improvement and Capability Determination (SPICE), a set of technical standards documents for the computer software development process and related business management functions.

The ARC Class A appraisals is normally conducted by SCAMPI A appraisal. The SCAMPI A Method Definition Document is where is defined rules to ensure the consistency of the appraisal ratings, so the same maturity rated in two companies means they are equal in methodologies and business processes.


\subsection{SCAMPI principles}
As said before the class A appraisal is the only full comprehensive appraisal method that involves an ARC class A method and uses CMMI models as reference models.

This appraisal will allow organizations to gain insight about their capability by identifying the strengths and weaknesses of its current processes, prioritize improvement plans, focus on those improvements, correcting weakness that will generate risks, derive capability rating as a maturity level rating and identify risks relative to capability and maturity determinations.

This appraisal follows these principals:
\begin{itemize}
	\item Start with a process reference model.
	\item Use a defined appraisal method.
	\item Involve senior management as an appraisal sponsor.
	\item Observe strict confidentiality and non-attribution.
	\item Approach the appraisal collaboratively. (When SCAMPI is used for Supplier Selection or Process Monitoring modes, it may not be
	possible to use a collaborative appraisal approach.)
	\item Focus on the sponsors business objectives
\end{itemize}

\subsection{The SCAMPI process}

The Method Definition Document is a document that describes SCAMPI appraisal method, this document sets the key elements of appraisal planning and the rules of conduct. It is also included in this document the level of process tailoring permitted, qualifications of the team members, evidence requirements, how to scope the appraisal and more.

There are essentially three phases in the process:
\begin{itemize}
	\item Phase I - Plan and Prepare for Appraisal presented in the Figure \ref{fig:plan_appraisal}
	\item Phase II – Conduct Appraisal presented in the Figure \ref{fig:results_appraisal}
	\item Phase III – Report Results
\end{itemize}

The following graphs shows us these phases where the last one includes the results report phase.

\begin{figure}[h]
	\begin{center}
		\leavevmode
		\includegraphics[width=0.86\textwidth]{appraisal_activies}
		\caption{Plan and Prepare for Appraisal Activities}
		\label{fig:plan_appraisal}
	\end{center}
\end{figure}


\begin{figure}[h]
	\begin{center}
		\leavevmode
		\includegraphics[width=0.86\textwidth]{phase2and3}
		\caption{Conduct Appraisal Activities}
		\label{fig:results_appraisal}
	\end{center}
\end{figure}

\subsection{Special terms}
There are some terms to consider with special meaning, Organizational Unit (OU), Organizational Scope, Subgroup, Basic Unit, Support Function, Objective Evidence, Instantiation, Database of Objective Evidence, Practice Characterization.

Organizational Unit is the subject of an appraisal. Can be deployed one or more processes that have a consistent process context, operates in a coherent set of business objectives and is typically part of a larger organization. In a small organization, this unit can be the whole organization.

Basic Unit stands for a set of interrelated and managed resources that delivers products or services to a customer and usually works like planned. The plan is documented and specifies the services or products delivered or implemented, the funds, the future work and the work that is currently being done.

A collection of basic unit and support functions that represent practices used within and organizational unit is the Organizational scope.


A Subgroup is a cluster of basic units that are shared between similar process implementations and a common sampling factor alternatives.

Support Function is an organizational group that for a certain and well defined set of activities needed by other parts of the organizations provides products and/or services.


Objective Evidence (OE) are indicators of the implementation or institutionalization of model practices. Verifying practice implementation is the review of Objective Evidence to determine whether a practice is implemented within a basic unit, support function, and/or organization. Can be of two types: artifacts or affirmations.
The artifacts are a tangible form of evidence indicative of work being done, which is both the main output of a practical model or a consequence of the implementation of a model of practice.
Affirmation is an oral or written statement confirming or support the implementation (or lack of implementation) in a practical model
provided by the practice performers, provide through an interactive forum in which the evaluation team has control over the
interaction.
In certain cases for some practices, documents are accepted as artifacts even if they are not the primary desired result of practical practice.

Instantiation is the implementation of a model practice used in its context in the organizational unit boundaries.

\subsection{Practice characterization}
Practices Implementation Indicators (PII) are a proof of  a correct implementation of a certain CMMI Practice. When a Practice is performed will leave a mark or evidence of that operation, for example that evidence can be a document produced while the practice is performed.

Appraisers look for an objective evidence in order to make an assessment. There are three types of indicators presented in the Table \ref{tab:Indicator}.


\begin{table}[h]
	\centering
	\caption{Indicators Types}
	\begin{tabular}{|p{2cm}|p{7cm}|p{4cm}|}
		\hline Indicator Type & Description & Examples\\
		\hline Direct artifacts & The tangible outputs resulting directly 
		from implementation of a specific or 
		generic practice. An integral part of 
		verifying practice implementation. May 
		be explicitly stated or implied by the 
		practice statement or associated 
		informative material. & Typical work products listed 
		in reference model practices 
		
		Target products of an 
		Establish and Maintain specific practice 
		
		Documents, deliverable 
		products, training materials, 
		etc. \\ 
		\hline Indirect artifacts & Artifacts that are a consequence of 
		performing a specific or generic practice 
		or that substantiate its implementation, 
		but which are not the purpose for which 
		the practice is performed. This indicator 
		type is especially useful when there may 
		be doubts about whether the intent of the 
		practice has been met (e.g., an artifact 
		exists but there is no indication of where 
		it came from, who worked to develop it, 
		or how it is used).  & Typical work products listed 
		in reference model practices 
		
		Meeting minutes, review 
		results, status reports, 
		presentations, etc. 
		
		Performance measures \\ 
		\hline Affirmations & Oral or written statements confirming or 
		supporting implementation (or lack of 
		implementation) of a specific or generic 
		practice. These statements are usually 
		provided by the implementers of the 
		practice and/or internal or external 
		customers, but may also include other 
		stakeholders (e.g., managers and 
		suppliers). &  Instruments 
		
		Interviews 
		
		Presentations, 
		demonstrations, etc.\\ 
		\hline 
	\end{tabular}
	\label{tab:Indicator}
\end{table}


\newpage
After the collection of an evidence and properly examined is made a characterization of the extent to which Model practices are implemented. The model practices are characterized as described in the Table \ref{tab:characterizations}.
\newline

\begin{table}[h]
	\centering
	\caption{Practice characterization table}
	\begin{tabular}{|p{4cm}|p{9cm}|}
		\hline
		Fully Implemented (FI)   & Sufficient artifacts and/or affirmations are present and
		judged to be adequate to demonstrate practice implementation, and
		no weaknesses are noted.    \\
		\hline
		Largely Implemented (LI) & Sufficient artifacts and/or affirmations are present and
		judged to be adequate to demonstrate practice implementation, and
		one or more weaknesses are noted.  \\ 
		\hline
		Partially Implemented (PI) & Some or all data required are absent or judged to be
		inadequate,
		Some data are present to suggest some aspects of the practice are
		implemented, and
		one or more weaknesses are noted.
		
		
		OR
		
		
		Data supplied to the team (artifacts and/or affirmations) conflict –some data
		indicate the practice is implemented and some data indicate the practice is
		not implemented, and
		one or more weaknesses are noted.\\
		\hline
		Not Implemented (NI) & Some or all data required are absent or judged to be
		inadequate,
		Data supplied does not support the conclusion that the practice is
		implemented, and
		one or more weaknesses are noted. \\
		\hline
		Not Yet (NY) & The basic unit or support function has not yet reached the stage in the
		sequence of work, or point in time to have implemented the practice. \\
		\hline
	\end{tabular}
	\label{tab:characterizations}
\end{table}





\subsection{Appraisal participants}
In an appraisal there are several participants with roles and responsibilities crucial to its success.

The Appraisal sponsor is responsible to sponsor the appraisal and owns the appraisal results and signs the Appraisal Disclosure Statement.

Middle managers are originally from the line or staff management positions and are interviewees and data providers and if they are participant they review preliminary findings.

Basic Unit leaders have leadership responsibilities for a project, service. They are too interviewees and data providers and if they are participant they review preliminary findings too.

Support Function as the past roles are interviewees and data providers, they are practitioners and review preliminary findings.

\subsection{Appraisal team}

The appraisal team is composed by two main Key Roles: Team Leader and Team Members.
Team Leader is the person who has the overall responsibility for the appraisal, is a SEI - Certified SCAMPI\citep{SCAMPITeam2013} leader appraisal and has experience and training, he signs too the final findings.
Team members are those who satisfy requirements of  experience and training to be part of the team and they assume one or more specific roles.

One of the key roles of the appraisal team is the team leader who has overall responsibility for the appraisal. He is also responsible for assign team roles for each member, ensuring that the planning activities are complete, that the SCAMPI process is being followed, scheduling monitoring and checking performance, facilitate team resolution in case of conflicts and impasses and reporting results to SEI.

For each team member the team leader will assign a role that will ensure the proper function of the team and will facilitate the appraisal, those roles are the following:
\begin{itemize}
	\item Appraisal coordinator
	\subitem Responsible for handling on-site logistics. This position is also composed by more than one member for a multi-site appraisal.
	\item Librarian
	\subitem Documents are managed by this member and in the end of the appraisal they are returned.
	\item Timekeeper
	\subitem For each mini-team can be one Timekeeper and his main purpose is track team time and schedule constraints during interviews and other activities.
	\item Note takers
	\subitem For all PAs is responsible for taking notes during data gathering sessions.
	\item Appraisal team
	\subitem All the work is reviewed by members.
	\item Mini-teams
	\subitem Teams typically consist of two or three members and verify the implementation of reference model practices, reviewing objective evidence provided and identify weaknesses in the implementation. The practices at instantiation levels are characterized by its implementation extent. They have the power to request addition information if needed. 
\end{itemize}

\subsection{SCAMPI results}

The appraisal is completed after the collection and evaluation of objective evidence to support the implementation of practices.

Goal satisfaction depends on satisfaction of practices associated with it.

The goal is rated if and only if all associated practices are characterized as largely implemented or fully implemented, and all the weaknesses associated with the defined goal don't have a significant impact on goal achievement.
With the help of a program we can obtain a matrix as shown in the Figure \ref{fig:scampiresults}.

% Following the collection and evaluation of objective evidence supporting the implementation of Practices, the extent to which the Goals (of a given CMMi Process Area) are being met can be accessed. 
% 
% Goal satisfaction is based on the extent to which the practices associated with that goal have been implemented. This determination is made using the concepts outlined in the previous section. 


\begin{figure}[h]
	\begin{center}
		\leavevmode
		\includegraphics[width=0.86\textwidth]{thesis_goals}
		\caption{SCAMPI results}
		\label{fig:scampiresults}
	\end{center}
\end{figure}


When a given Goal is determined to be either Satisfied or not, then a Capability Level (for the continuous representation) can be derived and its possible to appraise.

\section{SCRAIM}

SCRAIM \citep{SCRAIM} is a project management tool developed at Strongstep based on advanced methodologies with intelligent decision support mechanisms. It also has some ready-made processes that facilitate a better management.

\subsection{Software-as-a-Service}

SCRAIM is a SaaS, which stands for Software-as-a-Service. Software-as-a-service (SaaS) emerges as an innovative
approach to deliver software applications based on cloud-computing
technology. \citep{Chou:2007:ANI:1359479.1359484}

This type of software sometimes refered as simply hosted applications allows organizations and clients to access functionalities and all data stored on that platform everywhere, and it costs less than a typical licensed application. SaaS has many advantages compared to typical software, since is hosted remotely and accessed through Web they bypass server provisioning and software
installation as requirement, making software cheaper. Another advantage of this development type is that organizations don't need to perform and handle installation problems, updates and performing maintenance.

\begin{quote}
	``SaaS is one of the biggest technology trends to affect business
	applications in recent years.''~\cite{House2009}
\end{quote}
%Citar - SaaS is one of the biggest technology trends to affect business
%applications in recent years. (documento do SaaS)

\subsection{Methodologies and Processes definition}
One of the reasons that can lead to a project failure is the lack of use of a defined software development process, is also known that one of the success factors is the adoption of appropriate development process to the organizations projects.

\begin{figure}[h]
	\begin{center}
		\leavevmode
		\includegraphics[width=0.5\textwidth]{ScraimProcessChoose}
		\caption{SCRAIM Process choose wizard}
		\label{fig:scraimprocesschoose}
	\end{center}
\end{figure}

SCRAIM supports the most advanced technologies like CMMI \citep{Development2010}, TSP and SixSigma \citep{SixSigmaWeb} to help organizations increase projects quality.
SCRAIM has a set of ready made processes like SCRUM \citep{Pries2011}, chosen in the Figure \ref{fig:scraimprocesschoose} and its possible to adapt to the specific needs of a project and save to use it later.

\subsection{Project planning and tracking}

The Planning page represented by the Figure \ref{fig:scraimplanning} presents to the user a chronological view of the project's iterations (Centralized on the current iteration).

\begin{figure}[h]
	\begin{center}
		\leavevmode
		\includegraphics[width=0.86\textwidth]{ScraimPlanning2}
		\caption{SCRAIM Planning Page}
		\label{fig:scraimplanning}
	\end{center}
\end{figure}

In this page users are able to create, update and estimate tasks and assign those tasks to iterations and team members.

%referenciar

\subsection{Risk and issue management}

Risk Management is also supported by SCRAIM. This part of the software is designed to give the possibility to identify what can go wrong (risks), how to prevent that from happening (mitigation and contingency actions) and what to do if something happens (Impediments).

\begin{figure}[h]
	\begin{center}
		\leavevmode
		\includegraphics[width=0.86\textwidth]{scraimrisk}
		\caption{SCRAIM Risk View}
		\label{fig:scraimrisk}
	\end{center}
\end{figure}

\subsection{Test management}
Test management tools provided on SCRAIM are used to store information on how testing is to be done, plan testing activities and report the status of quality assurance activities.

In the Figure \ref{fig:scraimtest} is presented the project tests configuration, presenting all the actions that can be made by the user, and that are related to traceability and execution of test cases.

\begin{figure}[h]
	\begin{center}
		\leavevmode
		\includegraphics[width=0.86\textwidth]{scraimtest}
		\caption{SCRAIM Test configuration view}
		\label{fig:scraimtest}
	\end{center}
\end{figure}


\subsection{Other functionalities}
The project information is easily trackable, SCRAIM allows  to manage files and documentation associated to each one of the projects and to attach external repositories in order to track the changes of source code.

In SCRAIM its possible to schedule deliverables for each release of the organization project, with this its possible to know in real time what's being delivered, what's being schedule and who's in charge of each deliverable.

Wiki, forums, news and notification system are other set of features that facilitate team collaboration.