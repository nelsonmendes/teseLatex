\chapter{Conclusions and future work} \label{chap:conclusion}

\section*{}
In this chapter it is presented the summary of all the work done, the contributions of this thesis  and the future work.

\section{Achievements}

All the goals established in the start of this thesis were almost completely fulfilled. The group of tools, methodologies and techniques accomplished resulted in a prototype of an automatic assessment module in SCRAIM. The results generated by the prototype are promising, getting very close to a real assessment, as presented in Chapter \ref{chap:usage}. Despite the success of this tool there is still much work to be done in order to make it more usable and more embracing. This is explained in more detail in Section \ref{futurework}.

The greatest difficulty in this thesis was to understand and be able to apply the concepts and methodologies behind the CMMI in order to be able to map its practices to SCRAIM. This was due to the fact that my knowledge about CMMI was very limited. To overcome this limitation it was necessary a very intensive research and extensive study in this area, which led me to acquire interest for Software Engineering. The result of this research is presented in Chapter \ref{chap:problem}.

One of the objectives of this thesis was to understand the level of support of SCRAIM in the CMMI scenario; this is researched and presented in the Chapter \ref{chap:conception}, where it is made an assessment of tool using with another tool that is currently used by appraisers. In the same chapter, it is shown the results of that assessment and from that we can conclude that SCRAIM is not yet ready to support a high level of maturity.

Despite the high level of coverage of SCRAIM on CMMI practices, a fully automatic electronic assessment is not feasible, so to complement it is necessary to take a survey in order to answer some practices needs. 

Chapter \ref{chap:implementation} presented the architecture of the prototype created; that was another point of work, in order to get a flexible and scalable architecture and prototype.

A verification and validation of this work was done by presenting an usage example and performing a comparative analysis of a manual assessment and an automatic assessment in Chapter \ref{chap:usage}. 

Initially, it was intended to reduce the costs and time of a SCAMPI evaluation and with the comparative analysis made we can say that this approach will make that happen and the prototype when extended and completed with the future work will facilitate the SCAMPI appraisals.

\section{Future Work}\label{futurework}

The automatic assessment module is a tool that will facilitate the SCAMPI appraisals, but in order to satisfy completely the demands of an appraisal first of all it is necessary to implement all the rules established for all the process areas.

This module is fully capable of being extended so CMMI for Development is only a start point; we can add other assessment techniques like ITMARK and even add more rules and practices to SCRAIM in order to achieve more maturity levels.

In order to satisfy and help the appraisers in their field, another interesting addition is implement a feature where they can run one assessment and get some results automatically and then override the obtained results exporting all the information and generated charts.

One last and crucial thing to do is continuously improve the questions on the survey; with that we can get more accurate and precise results of the practices that we can't obtain automatically. 