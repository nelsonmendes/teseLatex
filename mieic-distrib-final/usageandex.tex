\chapter{Usage and experimentation} \label{chap:usage}

\section*{}

In this chapter is presented an usage example of the created prototype of this set of tools and methods. Is presented too a comparison between a manual evaluation and an evaluation generated by this tool.

\section{Usage example} \label{sec:usageexample}
	%Usage example - Dar um exemplo de utilização com printscrn

To demonstrate the prototype is presented in this section an usage example of the created methodologies and rules.

As said before all of this was implemented inside Scraim, so as a prerequisite is only possible to use this if you have an account on Scraim with this feature enabled.

\vspace{10 mm}

\textbf{Home Screen}

So if an user that is registered in Scraim is logged in his account, on the side bar of Scraim (its menu) is possible to see the assessment logo and if we click on that we will be guided to the assessments module homepage, presented in the Figure \ref{fig:no_assessments}.

\begin{figure}[h]
	\begin{center}
		\leavevmode
		\includegraphics[width=0.9\textwidth]{no_assessments}
		\caption{Homepage without assessments done}
		\label{fig:no_assessments}
	\end{center}
\end{figure}

This screen is shown when we don't have assessments performed. If there are some done in this screen is presented a list of the assessments, like in the Figure \ref{fig:done_assessments}. In the Image, the three assessments are shown by ascending date order, so on bot are the most recent assessments.

\begin{figure}[h]
	\begin{center}
		\leavevmode
		\includegraphics[width=0.9\textwidth]{done_assessments}
		\caption{Homepage with three assessments done}
		\label{fig:done_assessments}
	\end{center}
\end{figure}


\vspace{10 mm}

\textbf{Assessment Options}

In the Figure \ref{fig:done_assessments}, on the right top corner there is a button that says "New assessment". When that button is pressed the application leads us to a screen where is prepared the assessment. This preparation screen can be seen in the Figure \ref{fig:prepare_assessment}, in this page is needed to specify the name of the assessment if we want a custom name for the assessment. If the name is not specified a automatic name will be generated with the performed time and date of the assessment. Is needed too choose the project or projects that are going to be evaluated.

\begin{figure}[h]
	\begin{center}
		\leavevmode
		\includegraphics[width=0.9\textwidth]{prepare_assessment}
		\caption{Homepage without assessments done}
		\label{fig:prepare_assessment}
	\end{center}
\end{figure}

Only the text field can be empty, is mandatory to choose at least one project. After completing this process, the button make assessment can be clicked and that will lead us to the screen presented in the Figure \ref{fig:goto_survey}.

\vspace{10 mm}

\textbf{Survey Screen}

\begin{figure}[h]
	\begin{center}
		\leavevmode
		\includegraphics[width=0.9\textwidth]{goto_survey}
		\caption{After Automatic Assessment, needed Survey}
		\label{fig:goto_survey}
	\end{center}
\end{figure}

The Survey is the Screen where is needed to answer all the questions, none can be skipped and only after that we can have a full assessment done and a proper result.

\vspace{10 mm}

\textbf{Results}

When all the process is completed we can see the results obtained, the results can be viewed in the Figure \ref{fig:area_view}. In this screen we can see the result of a project per area, in this case is only contemplated the areas Process Monitoring and Control and Project Planning.

\begin{figure}[h]
	\begin{center}
		\leavevmode
		\includegraphics[width=0.9\textwidth]{area_view}
		\caption{Result of assessment, area view}
		\label{fig:area_view}
	\end{center}
\end{figure}

When inside the graph a certain area is clicked, like for example PP which stands for Project Planning, the content of the graph changes to the practices results, that can be checked in the Figure \ref{fig:practices_view}.


\begin{figure}[h]
	\begin{center}
		\leavevmode
		\includegraphics[width=0.9\textwidth]{practices_view}
		\caption{Result of assessment, practices view}
		\label{fig:practices_view}
	\end{center}
\end{figure}

Is possible to see more in detail the assessment result if the mouse cursor is over the bar that correspond to a practice, when that bar is clicked, is shown in the page more info about that practice. The Figure \ref{fig:practice_click}, shows us the information appended to the page when the Practice 1.1 of the first goal of Project Planning Area is clicked.

\begin{figure}[h]
	\begin{center}
		\leavevmode
		\includegraphics[width=0.9\textwidth]{practice_click}
		\caption{Result of assessment, practice view details}
		\label{fig:practice_click}
	\end{center}
\end{figure}

After all this process, all views are able to return to the Home screen, that contains the list of the assessments done, the assessment that we have done and we are seeing is already present on the list of assessments.

\section{Coverage percentage} \label{sec:coverage}

%Scraim extension to increase cmmi coverage

%Implementação até onde está neste momento falar das duas àreas totalmente implmentadas - PP e PMC

Despite the conception and the mapping of all Areas from second maturity level of CMMI for development, this prototype of the set of tools and methodologies only contemplates the implementation of two areas on the tool.

The Implemented areas are Project Planning and Project Monitoring and Control, those areas are the areas that is possible to get more feedback from the automatic part of the assessment. 
 
\section{Comparison between Automatic and Manual assessment} \label{sec:automatic}
%	Avaliação manual vs automatica trocar titulo

To determine if the implemented module is close to a real assessment is necessary to compare an automatic assessment (tool) and a manual assessment (human).

The chosen project is already finished and instantiated on SCRAIM.
The Manual assessment that is shown in Figure \ref{fig:manual_assessment} was performed by a Consultant from Strongstep.

\begin{figure}[h]
	\begin{center}
		\leavevmode
		\includegraphics[width=0.9\textwidth]{manual_assessment}
		\caption{Manual assessment, done for one appraisal}
		\label{fig:manual_assessment}
	\end{center}
\end{figure}

The Automatic assessment performed by the developed module can be seen in Figure \ref{fig:automatic_assessment}

\begin{figure}[h]
	\begin{center}
		\leavevmode
		\includegraphics[width=0.9\textwidth]{manual_assessment}
		\caption{Scraim Automatic Assessment}
		\label{fig:automatic_assessment}
	\end{center}
\end{figure}

In both assessments we can see that only the two areas featured on SCRAIM are object of this comparison.