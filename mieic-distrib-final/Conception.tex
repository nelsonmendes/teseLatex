\chapter{Conception} \label{chap:conception}

\section*{}

In this chapter is clarified and presented the scope of the project and the exploration of the solution.

\section{Approach} \label{sec:Approach}
In order to get a full overview of the Scraim, an assessment of the tool was needed. That assessment provides the gaps and the current state of the tool.
After analyzing the gaps a map can be done, that map will be done between CMMI for development practices and the tool.

\subsection{Scraim assessment}
In a primary phase as mentioned before, an assessment of the tool was performed. The actual purpose of this assessment was try to see if it was possible and currently viable match Scraim and its functionalities with the third maturity level of CMMI for Development. 

Maturity levels are a predefined set of process areas, measured by the respective goals of those areas and their practices. So if Scraim could be mapped to a more extensive number of practices and goals, a higher level of maturity could be covered.

The assessment was done with Appraisal Assistant, a tool currently used to assist and help appraisals in the field. This tool will allow to show the results in a matrix providing a full overview of the current state and the coverage of Scraim in relation to the maturity level number 3 of CMMI for Development.

The result of that assessment is shown in the matrix below.

Figure….
	
In the figure xx we can see that despite Scraim covers many practices, the maturity level three is still too far from being achieved successfully.
(falar dos exemplos de areas que nao podem ser cobertas)

The maturity level 2 is chosen to make a more precise and accurate coverage of CMMI practices and be able to map.



\subsection{Map CMMI - Scraim} \label{sec:mapping}
\subsubsection{Gap Analysis and Workaround}
	Mapping - apresentar as lacunas, e workaround
\subsubsection{Evaluation grids}
	Grelhas de avaliação e escala, explicando as escalas

\section{Scope}