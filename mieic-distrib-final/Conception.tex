\chapter{Conception} \label{chap:conception}

\section*{}

In this chapter is clarified and presented the scope of the project and the exploration of the solution.

\section{Approach} \label{sec:Approach}
In order to get a full overview of the Scraim, an assessment of the tool was performed. That assessment provides the current state of the tool and possible gaps to workaround.
After analyzing the gaps a map can be done, that map will be done between CMMI for development practices and the tool.

\subsection{Scraim assessment}
In a primary phase as mentioned before, an assessment of the tool was performed. The actual purpose of this assessment was try to see if it was possible and currently viable match Scraim and its functionalities with the third maturity level of CMMI for Development.

Maturity levels are a predefined set of process areas, measured by the respective goals of those areas and their practices. So if Scraim could be mapped to a more extensive number of practices and goals, a higher level of maturity could be covered.

The assessment was done with Appraisal Assistant, a tool currently used to assist and help appraisals in the field. This tool will allow to show the results in a matrix providing a full overview of the current state and the coverage of Scraim in relation to the maturity level number 3 of CMMI for Development.


\begin{figure}[h]
	\begin{center}
		\leavevmode
		\includegraphics[width=0.6\textwidth]{Level_3}
		\caption{Scraim Assessment to CMMI for Development Level 3}
		\label{fig:level3}
	\end{center}
\end{figure}

In the Figure \ref{fig:level3} we can see that despite Scraim covers many practices, the maturity level three is still too far from being achieved successfully.

For example some areas like Verification (VER) and Decision Analysis and Resolution (DAR), are totally impossible to achieve some type of automatic assessment.

The maturity level 2 is chosen to make a more precise and accurate coverage of CMMI practices and be able to map.

In the Figure \ref{fig:level2} we can see that the map for the second level of maturity is more accurate, more trustful and can be more covered automatically.
	
\begin{figure}[h]
	\begin{center}
		\leavevmode
		\includegraphics[width=0.6\textwidth]{Level_2}
		\caption{Scraim Assessment to CMMI for Development Level 2}
		\label{fig:level2}
	\end{center}
\end{figure}

The previous decision of making this set of tools to Scraim focusing in the level two is supported by this two assessments to the tool.



\subsection{Map CMMI - Scraim} \label{sec:mapping}
In the CMMI-SCRAIM map is where it has been evaluated the data model items and the level of support available on SCRAIM and match them to CMMI best practices.

Taking as reference the previous assessments of the tool, this map was possible with the creation of a new scale to show the results of the assessment and be possible the map corresponding the information.

One appraisal consists in classifying each practice in one of four states: Fully Implemented, Largely Implemented, Partially implemented and Not implemented.

\begin{figure}[h]
	\begin{center}
		\leavevmode
		\includegraphics[width=0.6\textwidth]{mapping_5levels}
		\caption{Project Planning SP1.1 Map to 5 levels}
		\label{fig:mapping_5levels}
	\end{center}
\end{figure}

In the first attempt, was tried to match those levels with 5 levels, corresponding to different evaluation levels in Scraim. In the Figure \ref{fig:mapping_5levels} is possible to see the map between CMMI for Development and Scraim. In this example is analyzed the criteria from the practice 1.1 from the Project Planning area, which says, "Establish a top-level work breakdown structure (WBS) to estimate the scope of the project." and can be mapped into Scraim into this 5 levels. The first Level would be where there is none WBS item or backlog item, and the max level where all WBS are fully defined and epics with backlog items associated. 

\begin{figure}[h]
	\begin{center}
		\leavevmode
		\includegraphics[width=0.6\textwidth]{mapping_4levels}
		\caption{Project Planning SP1.1 Map to 4 levels}
		\label{fig:mapping_4levels}
	\end{center}
\end{figure}

After some mapping this scale was confusing and far from real so the best way was to map and try match the four levels of SCAMPI with four levels of Scraim assessment.

This new scale is more accurate and closer to a real assessment. In the Figure \ref{fig:mapping_4levels} is observable that the mid levels are more easily distinguished and more differentiable from the others. 

\subsubsection{Gap Analysis and Workaround}
Regarding the mapping, it was discovered  that some practices can not be automatically evaluated, but with some extra work and little tweaks it is possible to solve those gaps and make an automatic assessment to evaluate such practices.

The gaps found can be solved with the integration of some Plugins, some frameworks and some rules of usage.

Rules, frameworks and Plugins:
\begin{itemize}
	\item Plugin for wiki templates
	
	This plugin allows to choose a wiki template when is added a new page. It is possible see a preview of the template before choose it.
	This plugin will allow to resolve and input information directly in Scraim, without the use of other programs to generate those documents.
	
	\item Plugin for Document Management System Features
	
	Allows to manage documents submitted on Scraim, document approval workflow to be configurable and maintain a version control of this documents.
	
	\item Certain names for submitted documents
	
	The Project plan must be submitted with a certain name to the scraim files, to be automatic evaluated.
	
\end{itemize}