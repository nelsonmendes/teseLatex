\chapter*{Resumo}



\chapter*{Abstract}

Due to the change of the paradigm of current markets, resulting from the phenomenon of globalization, organizations are forced to streamline their business, in order to be able to maintain their competitiveness and ensure a favorable market position. To achieve such agility, their processes need to be less time consuming and more effortless, so they can focus on what really matters: value creation.

Certifications are a formal recognition of an organization that will provide guidance and tools for those who want to ensure that their products and services consistently meet customer's requirements, and that quality is consistently improved. However useful for the organization, the evaluation for certification takes too much effort and time. For example, the SCAMPI method takes a significant effort, being in some cases a very painful and expensive process. SCAMPI is the Standard CMMI Appraisal Method for Process Improvement, the evaluation method of CMMI model. CMMI is a model for organizations to improve their processes and is required by many U.S. Government contracts, especially in software development. 
Tool support is fundamental for facilitating the adoption of CMMI practices. SCRAIM is an example of a project life cycle management tool specifically designed to facilitate CMMI implementations.

The main goal is develop a group of methodologies, techniques and tools integrated in the SCRAIM interface, that will make evaluations and certain parts of certifications easier and less painful for the SCRAIM users.

Although there are a number of life cycle and project management tools, few combine this with process management techniques. SCRAIM combines the two and will provide the users new features that will semi- automate the assessment for certification of an organization. The full automated process is not yet feasible, so human intervention is still mandatory. With the use of SCRAIM, good practices are followed and in the end the generated information facilitate the decision making process.

We can see many advantages of this innovation, and we believe that the application of this innovation will help reducing the costs and time of one evaluation using the SCAMPI method.