\chapter*{Resumo}
Com o efeito da globalização, assistimos a uma mudança no paradigma dos mercados atuais, mudança essa que forçou as organizações a melhorar os seus processos de modo a manter a sua competitividade e deste modo garantir uma posição favorável no mercado. Para que tal aconteça, é necessário que comecem a simplificar e a perder menos tempo nos processos, pois só assim é possível à organização focar-se no que realmente interessa: a criação de valor.

As certificações são um reconhecimento formal de uma organização que fornecem orientação e ferramentas, que garantem que os produtos e serviços vão de encontro às necessidades e requisitos dos clientes e a sua qualidade é constantemente melhorada. Apesar de serem úteis para a organização, as certificações exigem um esforço muito elevado e consomem muito tempo. Por exemplo, o método SCAMPI exige um esforço elevado, sendo por vezes muito doloroso e custoso monetariamente para a organização.

O método SCAMPI é o método standard de avaliação para a melhoria de processos, método de avaliação do modelo CMMI.

O CMMI é um modelo para as organizações melhorarem os seus processos e é exigido por muitos contratos do governo dos EUA, especialmente no âmbito do desenvolvimento de software. SCRAIM é a ferramenta que vai fornecer meios para simplificar esse tipo de avaliações, com o objetivo de economizar tempo e consequentemente dinheiro.

O objetivo principal é desenvolver um grupo de metodologias, técnicas e ferramentas integradas no SCRAIM, que irão tornar as avaliações e certas partes das certificações mais fáceis e menos extenuantes para os utilizadores do SCRAIM.
Embora haja um número elevado de ferramentas que permitem gerir os projetos e o ciclo de vida deles, poucas combinam isso com técnicas de gestão de processos. O SCRAIM combina os dois e irá fornecer funcionalidades que irão permitir semi-automatizar a avaliação para a certificação de uma organização. O processo totalmente automatizado ainda não é viável, pois a intervenção humana ainda é obrigatória.

Com o uso do SCRAIM, as boas práticas irão ser seguidas e, no final, a informação gerada irá facilitar no processo de tomada de decisão.

Podemos ver muitas vantagens da criação e desenvolvimento desta inovação e acreditamos que a sua aplicação ajudará a reduzir os custos e o tempo de uma avaliação utilizando o método SCAMPI.




\chapter*{Abstract}

Due to the change of the paradigm of current markets, resulting from the phenomenon of globalization, organizations are forced to streamline their business, in order to be able to maintain their competitiveness and ensure a favorable market position. To achieve such agility, their processes need to be less time consuming and more effortless, so they can focus on what really matters: value creation.

Certifications are a formal recognition of an organization that will provide guidance and tools for those who want to ensure that their products and services consistently meet customer's requirements, and that quality is consistently improved. However useful for the organization, the evaluation for certification takes too much effort and time. For example, the SCAMPI method takes a significant effort, being in some cases a very painful and expensive process. SCAMPI is the Standard CMMI Appraisal Method for Process Improvement, the evaluation method of CMMI model. CMMI is a model for organizations to improve their processes and is required by many U.S. Government contracts, especially in software development. 
Tool support is fundamental for facilitating the adoption of CMMI practices. SCRAIM is an example of a project life cycle management tool specifically designed to facilitate CMMI implementations.

The main goal is develop a group of methodologies, techniques and tools integrated in the SCRAIM interface, that will make evaluations and certain parts of certifications easier and less painful for the SCRAIM users.

Although there are a number of life cycle and project management tools, few combine this with process management techniques. SCRAIM combines the two and will provide the users new features that will semi- automate the assessment for certification of an organization. The full automated process is not yet feasible, so human intervention is still mandatory. With the use of SCRAIM, good practices are followed and in the end the generated information facilitate the decision making process.

We can see many advantages of this innovation, and we believe that the application of this innovation will help reducing the costs and time of one evaluation using the SCAMPI method.